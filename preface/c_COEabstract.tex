% !Mode:: "TeX:UTF-8"
% 文字编码:UTF-8
%%%%%%%%%%%%%%%%%%%%%%%%%%%%%%%%%%%%%%%%%%%%%%%%%%%%%%%%%%%%%%%%%%%%%%%%%
%
%   LaTeX File for Doctor (Master) Thesis of Peking University
%   LaTeX + CJK     北京大学博士(硕士)论文模板
%   Based on Wang Lei's Template for THU
%   Version: 1.00
%   Last Update: 2005-05-25
%
%%%%%%%%%%%%%%%%%%%%%%%%%%%%%%%%%%%%%%%%%%%%%%%%%%%%%%%%%%%%%%%%%%%%%%%%%
%   Copyright 2004-2005  by  Ying Pan       (yeying_pan@yahoo.com.cn)
%%%%%%%%%%%%%%%%%%%%%%%%%%%%%%%%%%%%%%%%%%%%%%%%%%%%%%%%%%%%%%%%%%%%%%%%%
%%%%%%%%%%%%%%%%%%%%%%%%%%%%%%%%%%%%%%%%%%%%%%%%%%%%%%%%%%%%%%%%%%%%%%%%%
%
%   LaTeX File for Doctor (Master) Thesis of Tsinghua University
%   LaTeX + CJK     清华大学博士(硕士)论文模板
%   Based on Wang Tianshu's Template for XJTU
%   Version: 1.00
%   Last Update: 2003-09-12
%
%%%%%%%%%%%%%%%%%%%%%%%%%%%%%%%%%%%%%%%%%%%%%%%%%%%%%%%%%%%%%%%%%%%%%%%%%
%   Copyright 2002-2003  by  Lei Wang (BaconChina)       (bcpub@sina.com)
%%%%%%%%%%%%%%%%%%%%%%%%%%%%%%%%%%%%%%%%%%%%%%%%%%%%%%%%%%%%%%%%%%%%%%%%%

\newitemsep
\renewcommand{\labelenumi}{(\arabic{enumi})}
\cabstract{
\thispagestyle{plain}
随着网络带宽的提升以及视频编解码技术的进步,网络上的视频资源、实时视频等迅速占据了网络的主要流量。而移动智能设备的普及和4G、Wi-Fi等无线网络覆盖率的提高,改变了人们的通信方式,使得随时随地的高质量视频通话成为可能。
%近年来电视盒子进入家庭,结合互联网和大屏电视,改变了人们的娱乐方式,同时使得更高质量的视频通话成为以后的发展趋势。基于电视盒子、手机平台和无线网络媒介的视频通话,结合了高清视频体验和方便的网络接入,是实时视频传输技术一种很有价值的应用场景。
另一方面,高清实时视频对网络质量尤其是网络延迟和丢包的高度敏感,与无线网络参差不齐的网络质量之间的矛盾极大阻碍了视频质量的提升。本课题以上述基于无线网络的视频通话场景为切入点,研究无线网络下高质量实时视频传输的实现,解决了其中的视频码率自适应和差错保护两方面的关键算法问题,并实现了高质量视频通话系统。
首先,我们关注在无线网络带宽波动大、存在一定丢包和延迟的网络条件下,如何及时调整视频码率从而在避免拥塞的前提下最有效地利用网络带宽。与传统拥塞控制算法不同的是,应用于实时视频的拥塞控制算法对网络延迟有更高的要求,同时需要考虑视频数据的编解码特性。为此,我们提出一种基于排队延迟和控制论优化的拥塞控制算法,通过将传输过程中的网络延迟控制在目标值附近来实现码率调整,同时利用控制论模型优化码率调整的速度和平滑性,进而在控制网络延迟波动的前提下对带宽进行更加高效稳定的利用。
其次,针对无线网络丢包率较高,需要进行一定的差错控制,而传统差错保护算法往往引入较大额外延迟的问题,我们采用了基于扩展窗口的FEC冗余保护方案,通过编码块的扩展基积累实现不引入额外延迟前提下编解码效率的提高。同时针对此模型复杂度较高,无法根据视频数据非对称特点进行进一步的冗余分配优化问题,我们对该扩展窗口框架进行了建模分析和等效化简,将冗余分配问题抽象为非线性整数优化问题,并进一步通过贪心算法进行求解,得到最优化的冗余分配方案。利用我们的算法,可以在不引入额外延迟的前提下,提高同等冗余度下的FEC 冗余保护的效果。
最后,在上述研究的基础上,我们基于开源软件实现了一个针对无线网络环境,面向电视盒子的高清实时视频通话系统。电视盒子是结合大屏电视屏幕和丰富互联网资源的一种智能设备,利用电视盒子进行实时视频通话可以获得更加真实、高清的全新通信体验。我们在此系统中增加了新的拥塞控制算法和FEC冗余保护模块,并针对大屏电视盒子进行了界面优化。与传统实时视频应用相比,我们的软件能更好地适应无线网络特性,提供更高质量的实时视频通话。

%\quad \\ \quad \\ \quad \\ \quad \\ \quad \\ \quad \\ \quad \\ \quad \\ \quad \\ \quad \\ \quad \\ \quad
\vfill
% 当关键词与脚注(某某基金资助)在同一页时,需要人工增减“\quad \\”来保证“关键词放摘要页最下方”。另一种方法是采用\vfill,但是在package.tex中使用脚注控宏包footmisc时就不能再采用选项bottom,代价是每页的脚注底部可能不对齐。

% 当中文摘要较长,使得关键词在摘要的第二页时,采用\vfill即可使关键词位于页面底部。(与英文摘要中关键词的处理方法相同)

关键词:实时视频传输,无线网络,码率自适应,非对称差错保护
}

