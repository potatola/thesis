% !Mode:: "TeX:UTF-8"
% 文字编码:UTF-8
%%%%%%%%%%%%%%%%%%%%%%%%%%%%%%%%%%%%%%%%%%%%%%%%%%%%%%%%%%%%%%%%%%%%%%%%%
%
%   LaTeX File for Doctor (Master) Thesis of Peking University
%   LaTeX + CJK     北京大学博士(硕士)论文模板
%   Based on Wang Lei's Template for THU
%   Version: 1.00
%   Last Update: 2005-05-25
%
%%%%%%%%%%%%%%%%%%%%%%%%%%%%%%%%%%%%%%%%%%%%%%%%%%%%%%%%%%%%%%%%%%%%%%%%%
%   Copyright 2004-2005  by  Ying Pan       (yeying_pan@yahoo.com.cn)
%%%%%%%%%%%%%%%%%%%%%%%%%%%%%%%%%%%%%%%%%%%%%%%%%%%%%%%%%%%%%%%%%%%%%%%%%
%%%%%%%%%%%%%%%%%%%%%%%%%%%%%%%%%%%%%%%%%%%%%%%%%%%%%%%%%%%%%%%%%%%%%%%%%
%
%   LaTeX File for Doctor (Master) Thesis of Tsinghua University
%   LaTeX + CJK     清华大学博士(硕士)论文模板
%   Based on Wang Tianshu's Template for XJTU
%   Version: 1.00
%   Last Update: 2003-09-12
%
%%%%%%%%%%%%%%%%%%%%%%%%%%%%%%%%%%%%%%%%%%%%%%%%%%%%%%%%%%%%%%%%%%%%%%%%%
%   Copyright 2002-2003  by  Lei Wang (BaconChina)       (bcpub@sina.com)
%%%%%%%%%%%%%%%%%%%%%%%%%%%%%%%%%%%%%%%%%%%%%%%%%%%%%%%%%%%%%%%%%%%%%%%%%

\newitemsep
\renewcommand{\labelenumi}{(\arabic{enumi})}
\cabstract{
\thispagestyle{plain}
随着移动智能设备的发展和3G、4G、WiFi等无线网络的普及,实时视频通话的使用越来越广泛,通话质量大幅度提高。另一方面,实时视频对网络质量尤其是网络延迟的高度敏感和无线网络参差不齐的网络质量之间的矛盾极大阻碍了实时视频服务质量的提升。本次课题抓住逐渐增加的高清视频通话应用的需求,研究面向无线网络的高清实时视频通话系统的实现,并解决了其中的关键算法问题。
首先,我们关注在有限并且不断波动的网络带宽下,如何及时调整视频码率从而在避免拥塞的前提下最有效地利用网络带宽。与传统拥塞控制算法(TCP、TFRC等协议)不同的是,应用于实时视频的拥塞控制算法对网络延迟有更高的要求,同时需要考虑视频数据本身的特点。为此,我们提出了一个基于排队延迟和控制论优化的拥塞控制算法,在保证网络延迟低于目标值的前提下对带宽进行更加高效稳定的利用。
其次,针对无线网络较高丢包率,而传统差错保护算法往往引入较大额外延迟的问题,我们采用了基于扩展窗口的FEC冗余保护方案,并在此基础上结合实时视频数据包重要性不同的特点,对冗余数据的分配进行了优化。利用我们的优化框架,可以在不引入额外延迟的前提下,大大提高FEC冗余保护的效果。
最后,在上述研究的基础上,我们对一个开源VOIP软件Linphone进行了系统实现。增加了新的拥塞控制算法和FEC冗余保护模块,并针对大屏电视盒子进行了界面优化。与传统实时视频应用相比,我们的软件能更好地适应无线网络特性,提供更高质量的实时视频通话。

%\quad \\ \quad \\ \quad \\ \quad \\ \quad \\ \quad \\ \quad \\ \quad \\ \quad \\ \quad \\ \quad \\ \quad
\vfill
% 当关键词与脚注(某某基金资助)在同一页时,需要人工增减“\quad \\”来保证“关键词放摘要页最下方”。另一种方法是采用\vfill,但是在package.tex中使用脚注控宏包footmisc时就不能再采用选项bottom,代价是每页的脚注底部可能不对齐。

% 当中文摘要较长,使得关键词在摘要的第二页时,采用\vfill即可使关键词位于页面底部。(与英文摘要中关键词的处理方法相同)

关键词:实时视频传输,无线网络,码率自适应,非对称差错保护
}

