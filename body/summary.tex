% !Mode:: "TeX:UTF-8"
% 文字编码:UTF-8
\chapter{总结与展望}
\label{chap:conclusion}

\section{本文工作总结}
科技的进步不断改变人们的生活方式,给人们的生活带来便利,视频通话的兴起和发展就是一个很好的例证。当然任何技术的进步都离不开学术界和工业界的共同努力,视频通信服务的发展也并非一帆风顺。实时视频对丢包和延迟高度敏感以及要求较大且稳定的带宽,与无线网络质量多变、不可靠的传输环境相矛盾,给高质量的实时视频传输服务带来很大挑战。为了在无线网络下进行高质量的视频通话,可以从拥塞控制和差错控制两方面进行优化。针对动态的网络带宽,及时地调整传输视频的码率可以避免网络拥塞并最大化带宽利用率。另一方面,针对无线网络易丢包的特点以及实时视频对延迟的需求,采用FEC编码可以为视频流提供一定的抗丢包能力,获得更好的视频体验。本文针对上述两个方面分别进行了研究,我们的创新点主要包括以下方面:
\begin{enumerate}
    \item 针对无线网络中的实时视频传输,我们提出了一个分布式、延迟可控的码率自适应算法,满足了视频传输中的低延迟、带宽稳定、高带宽利用率等需求。我们首先通过对网络建模分析,提出了简化网络链路的排队延迟模型,并引入了影子价格和失真权重参数来更好地实现拥塞控制和多流公平。通过建立闭环反馈控制模型,并引入比例控制器进行参数优化,使得控制过程更加高效、平滑。另外,我们实现了一套完整的算法测试平台,并对算法进行了系统实现和大量效果测试。实验表明,相比于其他主流码率自适应算法,我们的算法在带宽利用率、稳定性,视频质量等方面都有更好的表现。
    \item 我们对一种针对实时视频传输的扩展窗口FEC框架进行了建模、分析,并提出了一种基于此框架的冗余分配方案。我们首先针对EW-RS框架进行了准确的丢包率分析,并针对其编解码特征提出了两条推论。然后我们推导出了GOP整体语预期失真公式,作为冗余最优化分配的理论基础。在此基础上,冗余分配问题被归纳为带约束的非线性优化问题。另外,为了简化这一问题的求解,我们设计了一种基于贪心的次优求解算法,使这一冗余分配算法能够适用于实时传输场景。大量实验结果表明,我们的算法在各种网络场景下都能明显提高视频传输的FEC保护效果。
    \item 利用本文中算法方面的工作,我们基于开源软件对齐内核进行了重写,实现了新的码率自适应和差错保护模块,实验表明这一新的底层模块大大改进了原软件的视频通话体验。为了充分发挥高质量视频传输的功能,我们还对其界面进行了优化,将其移至到电视盒子上,从而实现手机、电视双平台的高清视频通话。
\end{enumerate}


\section{未来工作展望}
在下一步的工作中,我们将继续深入实时视频传输过程中的拥塞控制和差错保护优化。尽管本文提出的算法已经在一定程度上改善了无线网络上实时视频传输的效果,但还远远无法满足日益增长的高清视频需求和日益多样化的网络环境。未来的工作可以从以下不同方面展开:
\begin{enumerate}
    \item 视频码率自适应算法可以参考的网络参数包括丢包率、延迟、抖动等,多数算法都只针对其中一个或几个方面进行了优化,却很难兼顾这些参数之间的关系及其对视频质量的系统性影响。例如只针对延迟进行调整,则算法在于基于丢包的算法竞争时存在饥饿现象;而只对网络丢包进行反应,则会造成网络延迟急剧增加。如何更好地综合各种网络参数,对网络状态进行估计进而控制视频码率,是下一步可以优化的问题之一。
    \item 视频马赛克、丢帧以及通话延迟等是公认影响视频通话用户体验的因素,然而还没有客观的标准定量地评价这些指标对用户体验的影响。这一方面给传输优化算法的比较带来困难,另一方面也不利于视频传输的进一步优化。因此在视频体验的定量评价方面也可以进行一些实验和研究。
    \item 在智能手机上进行FEC编解码会对其计算资源带来比较大的负载,引起发热、性能下降等问题。因此针对低性能的移动设备进行FEC编解码优化也是进一步提高实时视频通话体验的一种方法。
\end{enumerate}
