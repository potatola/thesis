% !Mode:: "TeX:UTF-8"
% 文字编码:UTF-8
%%%%%%%%%%%%%%%%%%%%%%%%%%%%%%%%%%%%%%%%%%%%%%%%%%%%%%%%%%%%%%%%%%%%%%%%%
%
%   LaTeX File for Doctor (Master) Thesis of Tsinghua University
%   LaTeX + CJK     清华大学博士(硕士)论文模板
%   Based on Wang Tianshu's Template for XJTU
%   Version: 1.51
%   Last Update: 2004-01-15
%
%%%%%%%%%%%%%%%%%%%%%%%%%%%%%%%%%%%%%%%%%%%%%%%%%%%%%%%%%%%%%%%%%%%%%%%%%
%   Copyright 2002-2004  by  Lei Wang (BaconChina)       (bcpub@sina.com)
%%%%%%%%%%%%%%%%%%%%%%%%%%%%%%%%%%%%%%%%%%%%%%%%%%%%%%%%%%%%%%%%%%%%%%%%%

%%%%%%%%%%%%%%%%%%%%%%%%%%%%%%%%%%%%%%%%%%%%%%%%%%%%%%%%%%%%%%%%%%%%%%%%%
%
%   LaTeX File for phd thesis of xi'an Jiao Tong University
%
%%%%%%%%%%%%%%%%%%%%%%%%%%%%%%%%%%%%%%%%%%%%%%%%%%%%%%%%%%%%%%%%%%%%%%%%%
%   Copyright 2002  by  Wang Tianshu    (tswang@asia.com)
%%%%%%%%%%%%%%%%%%%%%%%%%%%%%%%%%%%%%%%%%%%%%%%%%%%%%%%%%%%%%%%%%%%%%%%%%

%%%%%%%%%%%%%%%%%%%%%%%%%%%%%%%%%%%%%%%%%%%%%%%%%%%%%%%%%%%
%
% 引用的宏包和相应的定义
%
%%%%%%%%%%%%%%%%%%%%%%%%%%%%%%%%%%%%%%%%%%%%%%%%%%%%%%%%%%%

\usepackage{graphicx}
\usepackage[CJKbookmarks=true,
            bookmarksnumbered=true,
            bookmarksopen=true,
            colorlinks=false,
            citecolor=blue,
            linkcolor=red,
            anchorcolor=green,
            urlcolor=blue
            ]{hyperref}


% 伪代码
\usepackage{algorithm}
\usepackage{algpseudocode}
\floatname{algorithm}{算法}

% C++代码
\usepackage{listings}
\lstset{breaklines}%这条命令可以让LaTeX自动将长的代码行换行排版
\lstset{extendedchars=false}%这一条命令可以解决代码跨页时,章节标题,页眉等汉字不显示的问题
\lstset{
    keywordstyle=\bf\color{blue},   %code关键字红色
    commentstyle=\color{blue}, % 蓝色注释
    escapeinside=`'
}
% 子图的标题
\usepackage{subcaption}

% 支持彩色
\usepackage{color}

% 首行缩进宏包
\usepackage{indentfirst}

% 版面控制宏包,定义规定的版面尺寸
% \left = Word模版左侧页边距 = 2.6cm
% \right = Word模版右侧页边距 = 2.6cm
% \top = Word模版页边距(上)+ Word第一行文字上边缘距版心上边缘的距离 = 3cm + 0.15cm = 3.15cm
% \headsep = Word模版页边距(上) - Word模版页眉顶端距离 - Word模版一行页眉的高度 + Word第一行文字上边缘距版心上边缘的距离 = 3cm - 2cm  - 0.75cm + 0.15cm = 0.4cm
% 注:Word模版一行页眉的高度 略大于 Word模版页眉的行距(20pt)

% \bottom = Word模版页边距(下)= 2.5cm
% \footskip = Word模版页边距(下) - Word模版页脚底端距离 = 2.5cm - 1.75cm = 0.75cm

\usepackage[a4paper,left=2.6cm,right=2.6cm,top=3.15cm,headsep=0.4cm,headheight=0.75cm,,bottom=2.5cm,footskip=0.75cm,footnotesep=0.6cm]{geometry}



%数字转化为汉字
%\usepackage{zhnumber}


% AMSLaTeX宏包 用来排出更加漂亮的公式
\usepackage{amsmath}
\usepackage{amssymb}

%\newcommand\hmmax{0} % default 3
%\newcommand\bmmax{2} % default 4
% 处理数学公式中的黑斜体的宏包
\usepackage{bm}



% 不同于\mathcal or \mathfrak 之类的英文花体字体
%\usepackage{mathrsfs}

% 定理类环境宏包,其中 amsmath 选项用来兼容 AMS LaTeX 的宏包
\usepackage[amsmath,thmmarks]{ntheorem}

% 因为图形可浮动到当前页的顶部,所以它可能会出现
% 在它所在文本的前面. 要防止这种情况,可使用 flafter
% 宏包
%\usepackage{flafter}

%浮动图形控制宏包
%允许上一个section的浮动图形出现在下一个section的开始部分
%该宏包提供处理浮动对象的 \FloatBarrier 命令,使所有未处
%理的浮动图形立即被处理
\usepackage[below]{placeins}

% 图文混排用宏包
%\usepackage{floatflt}

% 图形和表格的控制
%\usepackage{rotating}

% tex1cm宏包,控制字体的大小
\usepackage{type1cm}

% 控制标题的宏包
%\usepackage{titlesec}

% 控制目录的宏包
\usepackage{titletoc}


%可将浮动对象放置到文件的最后
%\usepackage{endfloat}

% 脚注控制,每页脚注重新编号,每页脚注底部对齐
\usepackage[perpage,symbol*,bottom]{footmisc}

% fancyhdr宏包 页眉和页脚的相关定义
\usepackage{fancyhdr}
\usepackage{fancyref}

% 支持引用的宏包
\usepackage{cite}
% 支持引用缩写的宏包
\usepackage[numbers,sort&compress]{natbib}
%\usepackage{hypernat}

%浮动图形和表格标题样式
\usepackage{caption}

% 定制表格和图形的多行标题行距
\usepackage{setspace}

% 打印当前页面格式的宏包
%\usepackage{layouts}

% 使用Times字体的宏包
%\usepackage{times}
%\usepackage{mathptm}
%\usepackage[slantedGreek]{mathptmx}
\usepackage{mathptmx}
%\usepackage{txfonts}

% use url.sty
\usepackage{url}
% 使用跨页表格的宏包
%\usepackage{supertabular}
% 生成索引
\usepackage{makeidx}

%Harvard类型参考文献
%\usepackage{harvard}

%支持摄氏度等国际单位的宏包
\usepackage[squaren]{SIunits}

%处理item等环境的宏包,可用inparaenum等
%\usepackage{paralist}
%\usepackage{enumitem}

%表格合并多行
\usepackage{multirow}

%画备注框框需要的宏包
\usepackage{fancybox}

%显示latex页面设置的宏包
%\usepackage{showframe}

%pifont宏包含有\ding{}命令,可以在脚注使用带圈的数字,不过由于该字体只有1~10的符号,所以每页脚注不要超过10个。
\usepackage{pifont}

%pdfpages宏包可以将已有的pdf文件(例如版权声明和原创性声明)加入到文档中
\usepackage{pdfpages}

%\usepackage{threepartbox}%表格加注释

% 中文支持宏包
%\usepackage[boldfont,EmboldenFactor=2]{xeCJK}
\xeCJKsetup{boldfont,EmboldenFactor=2}
\usepackage{CJKnumb}
\usepackage{fontspec}
\setmainfont{Times New Roman}
\setsansfont{Arial}
\setCJKmainfont{SimSun}
\xeCJKsetwidth{[}{0.2em}
%\CJKsetecglue{}

\NeedsTeXFormat{LaTeX2e}

\DeclareMathAlphabet{\mathsfsl}{OT1}{cmss}{m}{sl}
%\newcommand{\tensor}[1]{\mathsfsl{#1}}
%\newcommand{\sub}[1]{_{_{\scriptstyle {#1}}}}
\newcommand{\tabincell}[2]{\begin{tabular}{@{}#1@{}}#2\end{tabular}}
